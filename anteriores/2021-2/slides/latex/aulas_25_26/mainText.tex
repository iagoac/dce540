\documentclass[compress]{beamer}
\usetheme{sthlm}

%-=-=-=-=-=-=-=-=-=-=-=-=-=-=-=-=-=-=-=-=-=-=-=-=
%        LOADING BEAMER PACKAGES
%-=-=-=-=-=-=-=-=-=-=-=-=-=-=-=-=-=-=-=-=-=-=-=-=

\usepackage{
booktabs,
datetime,
dtk-logos,
graphicx,
multicol,
pgfplots,
ragged2e,
tabularx,
tikz,
wasysym,
multirow,
float,
caption,
subcaption
}

\pgfplotsset{compat=1.8}

\usepackage[utf8]{inputenc}
\usepackage[portuguese]{babel}
\usepackage[T1]{fontenc}
\usepackage{newpxtext,newpxmath}
\usepackage{listings}

\lstset{ %
language=[LaTeX]TeX,
basicstyle=\normalsize\ttfamily,
keywordstyle=,
numbers=left,
numberstyle=\tiny\ttfamily,
stepnumber=1,
showspaces=false,
showstringspaces=false,
showtabs=false,
breaklines=true,
frame=tb,
framerule=0.5pt,
tabsize=4,
framexleftmargin=0.5em,
framexrightmargin=0.5em,
xleftmargin=0.5em,
xrightmargin=0.5em
}



%-=-=-=-=-=-=-=-=-=-=-=-=-=-=-=-=-=-=-=-=-=-=-=-=
%        LOADING TIKZ LIBRARIES
%-=-=-=-=-=-=-=-=-=-=-=-=-=-=-=-=-=-=-=-=-=-=-=-=

\usetikzlibrary{
backgrounds,
mindmap
}

%-=-=-=-=-=-=-=-=-=-=-=-=-=-=-=-=-=-=-=-=-=-=-=-=
%        BEAMER OPTIONS
%-=-=-=-=-=-=-=-=-=-=-=-=-=-=-=-=-=-=-=-=-=-=-=-=

\setbeameroption{show notes}

%-=-=-=-=-=-=-=-=-=-=-=-=-=-=-=-=-=-=-=-=-=-=-=-=
%        BEAMER COMMANDS
%-=-=-=-=-=-=-=-=-=-=-=-=-=-=-=-=-=-=-=-=-=-=-=-=


%-=-=-=-=-=-=-=-=-=-=-=-=-=-=-=-=-=-=-=-=-=-=-=-=
%
%	PRESENTATION INFORMATION
%
%-=-=-=-=-=-=-=-=-=-=-=-=-=-=-=-=-=-=-=-=-=-=-=-=

\title{Tolerância a \\ falhas}
\subtitle{DCE540 - Computação Paralela e Distribuída}
%\date{\small{\jobname}}
\author{\texttt{Iago Carvalho}}
\institute{\texttt{Departamento de Ciência da Computação}}

\hypersetup{
pdfauthor = {Iago A. Carvalho},      
pdfsubject = {Computação Paralela e Distribuída},
pdfkeywords = {},  
pdfmoddate= {D:\pdfdate},          
pdfcreator = {WriteLaTeX}
}

\begin{document}

\begin{frame}
\titlepage

\end{frame}

%% --------------------------------------------------------

\begin{frame}{Falhas}

Um sistema distribuído pode falhar

\begin{itemize}
    \item Falha em um componente
    \item Falha no processo de comunicação entre componentes
\end{itemize}

\vspace{0.5cm}

Falhas podem acontecer por diversos motivos

\begin{itemize}
    \item Quedas de energia
    \item Ataques externos
    \item Condições ambientais
\end{itemize}

\vspace{0.5cm}

Falhas podem tornar o sistema mais lento ou inutilizável

\begin{itemize}
    \item Gerar enormes perdas materiais
    \item Causar outros tipos de problemas
\end{itemize}

\end{frame}

%% --------------------------------------------------------

\begin{frame}{Conceitos básicos}

Devemos estabelecer alguns conceitos básicos para melhor entendermos o que é a tolerância a erros

\vspace{1cm}

Os quatro principais conceitos são
\begin{enumerate}
    \item Disponibilidade
    \item Confiabilidade
    \item Segurança
    \item Manutenibilidade
\end{enumerate}
\end{frame}

%% --------------------------------------------------------

\begin{frame}{Disponibilidade e confiabilidade}

\textbf{Disponibilidade}: O sistema tem que estar pronto para uso imediato
\begin{itemize}
    \item A qualquer momento, o sistema tem que estar disponível
    \item Capaz de funcionar normalmente
    \item Alta disponibilidade é esperada
\end{itemize}

\vspace{0.5cm}

\textbf{Confiabilidade}: O sistema tem que funcionar corretamente e continuamente, sem interrupções de serviço
\begin{itemize}
    \item Grandes janelas de tempo sem interrupções ou falhas
    \item Um sistema confiável tem poucas falhas ou interrupções
    \begin{itemize}
        \item Mesmo que estas interrupções sejam grandes
    \end{itemize}
\end{itemize}
\end{frame}

%% --------------------------------------------------------

\begin{frame}{Métrica de confiabilidade}

Pode-se medir a confiabilidade de um componente levando em consideração duas métricas
\begin{enumerate}
    \item Tempo médio para falha ($T_f$)
    \item Tempo médio para recuperação ($T_r$)
\end{enumerate}

\vspace{0.5cm}

$$
\textnormal{Confiabilidade} = \frac{T_f}{T_f + T_r}
$$

\vspace{0.5cm}

Ainda devemos considerar que o tempo médio para falha pode diminuir com o passar do tempo
\begin{itemize}
    \item Desgaste físico do componente
    \item Alterações no sistema distribuído que podem introduzir \textit{bugs}
\end{itemize}
\end{frame}
%% --------------------------------------------------------

\begin{frame}{Segurança e manutenibilidade}

\textbf{Segurança}: Quando um sistema falha temporariamente, nenhum evento catastrófico pode ocorrer
\begin{itemize}
    \item Exemplos de eventos catastróficos
    \begin{itemize}
        \item Falha do controle de temperatura de uma usina nuclear
        \item Sistemas controladores de tráfego aéreo deixam de monitorar certos aviões
        \item Sistemas de vigilância deixam de identificar ameaças
    \end{itemize}
    \item Sistemas seguros são aqueles que não ocorrem falhas catastróficas
\end{itemize}

\vspace{0.25cm}

\textbf{Manutenibilidade}: No caso de uma falha, o sistema tem que poder ser recuperado facilmente
\begin{itemize}
    \item Recuperação automática é desejada
    \item Recuperação rápida
    \item Relacionado a disponibilidade
    \begin{itemize}
        \item Sistema com auta manutenibilidade, em geral, tende a ter maior disponibilidade
    \end{itemize}
\end{itemize}

\end{frame}

%% --------------------------------------------------------

\begin{frame}{Falha, erro e falta}

Uma \textbf{falha} é quando um componente não consegue cumprir suas premissas
\begin{itemize}
    \item Quando o componente não consegue funcionar corretamente
    \item Ele não consegue prover seus serviços
\end{itemize}

\vspace{0.5cm}

Um \textbf{erro} é a parte do estado do sistema que levou a falha
\begin{itemize}
    \item Erros na transmissão de dados
    \item Erros no processo de criptografia
    \item $\ldots$
\end{itemize}

\vspace{0.5cm}

Uma \textbf{falta} é o que ocasionou o erro.
\begin{itemize}
    \item Meio de transmissão de dados comprometido
    \item Atualização parcial do algoritmo de criptografia
\end{itemize}
\end{frame}

%% --------------------------------------------------------

\begin{frame}{Tolerância a falhas}

Um componente (ou sistema) tolerante a falhas é aquele que consegue funcionar corretamente mesmo caso ocorram uma ou mais faltas

\vspace{0.5cm}

O componente é capaz de realizar suas tarefas mesmo com uma ou mais faltas
\begin{itemize}
    \item Por exemplo, aplicando protocolos de correção de erros
    \item Acessando dados replicados
    \item Caminhos alternativos para roteamento de mensagens
\end{itemize}

\end{frame}

%% --------------------------------------------------------

\begin{frame}{Periodicidade de falhas}

\textbf{Transiente}: A falha ocorre uma única vez
\begin{itemize}
    \item Mesmo que a ação que levou a falha seja repetida, ela não ocorre novamente
    \item Comum devido a eventos físicos
\end{itemize}

\vspace{0.35cm}

\textbf{Intermitente}: A falha se repete ocasionalmente
\begin{itemize}
    \item O sistema funciona corretamente em alguns momentos, e não em outros
    \item Difícil de se detectar e recuperar
\end{itemize}

\vspace{0.35cm}

\textbf{Permanente}: Falhas cuja única recuperação é a substituição do componente defeituoso
\begin{itemize}
    \item Um disco rígido queimado
    \item Bug de software
    \item Cabo de transmissão de dados rompido
\end{itemize}

\end{frame}

%% --------------------------------------------------------

\begin{frame}{Tipos de falhas}

% Please add the following required packages to your document preamble:
% \usepackage{graphicx}
% \usepackage[table,xcdraw]{xcolor}
% If you use beamer only pass "xcolor=table" option, i.e. \documentclass[xcolor=table]{beamer}
\begin{table}[]
\centering
\label{tab:my-table11}
\resizebox{\textwidth}{!}{%
\begin{tabular}{cc}
\hline
{\textbf{Tipo de falha}} & {\textbf{Comportamento do sistema}}  \\ \hline
Queda      & Para de funcionar                \\
Omissão    & Falha na troca de mensagens      \\
Timing     & Respostas são enviadas em atraso \\
Resposta   & Uma resposta incorreta é enviada \\
Arbitrária & Respostas arbitrárias são enviadas \\ \hline
\end{tabular}%
}
\end{table}

Omissão pode ser para receber ou enviar mensagems

\vspace{0.25cm}

Resposta também pode ter dois tipos
\begin{enumerate}
    \item Valor
    \item Fluxo de controle
\end{enumerate}
\end{frame}

%% --------------------------------------------------------

\begin{frame}{Diferenças entre sistemas síncronos e assíncronos}

\textbf{Assíncrono}: Se um componente $P$ deixa de receber mensagens de outro componente $Q$, ele não pode assumir que $Q$ falhou
\begin{itemize}
    \item Sua única opção é continuar esperando
    \item Diminuir o ritmo de envio de mensagens
\end{itemize}

\vspace{0.5cm}

\textbf{Síncrono}: Se $P$ deixa de receber mensagens de $Q$, pode-se efetivamente assumir uma falha no sistema distribuído
\begin{itemize}
    \item $Q$ falhou
    \item Caminho de $P$ para $Q$ falhou
    \begin{itemize}
        \item Faz-se necessário recalcular as rotas de comunicação
    \end{itemize}
\end{itemize}
\end{frame}

%% --------------------------------------------------------

\begin{frame}{Gravidade de falhas}

\textbf{Queda}: Componente $P$ pode efetivamente detectar que houve uma queda no componente $Q$

\vspace{0.5cm}

\textbf{Ruidosa}: Componente $P$ \textit{eventualmente} consegue detectar uma queda no componente $Q$

\vspace{0.5cm}

\textbf{Silenciosa}: Componente $P$ não consegue distinguir se houve uma queda em $Q$ ou se existe falhas de omissão ou timing

\vspace{0.5cm}

\textbf{Arbitrária}: Componente $P$ não consegue, efetivamente, detectar falha em $Q$. Mesmo se a falha for detectada, não é possível detectar o tipo da falha

\end{frame}

%% --------------------------------------------------------

\begin{frame}{Mascaramento de falhas utilizando redundância}

A melhor maneira de tornar um sistema (ou componente) tolerante a falhas é trabalharmos para tornar a falha transparente
\begin{itemize}
    \item Mascarar a falha de forma que o usuário não a perceba
\end{itemize}
    
\vspace{0.5cm}

Pode-se fazer isto efetivamente utilizando redundância
\begin{itemize}
    \item Redundância de informação
    \item Redundância de tempo
    \item Redundância física
\end{itemize}
\end{frame}

%% --------------------------------------------------------

\begin{frame}{Mascaramento de falhas utilizando redundância}

\textbf{Informação}: Informação extra é utilizada para detectar erros nas mensagens
\begin{itemize}
    \item Códigos de correção de erros
    \item Código de Hamming \href{https://pt.wikipedia.org/wiki/C\%C3\%B3digo\_de\_Hamming}{\beamergotobutton{Link}} 
\end{itemize}

\vspace{0.5cm}

\textbf{Tempo}: Caso uma resposta não chegue após um tempo predefinido, a comunicação é realizada novamente
\begin{itemize}
    \item Útil em sistemas síncronos
\end{itemize}

\vspace{0.5cm}

\textbf{Física}: Componentes adicionais são adicionados
\begin{itemize}
    \item Componente adicional é uma cópia exata de outro
    \item Alto custo
    \begin{itemize}
        \item Custo financeiro
        \item Custo computacional devido a replicação de dados
    \end{itemize}
\end{itemize}
\end{frame}

%% --------------------------------------------------------

\begin{frame}{Detecção de falhas utilizando grupos homogêneos}

Componentes são organizados em pequenos grupos

\vspace{0.5cm}

Em sua implementação mais simples, um grupo é composto por diversas replicações de um mesmo componente
\begin{itemize}
    \item Comunicações são realizadas entre grupos e não para componentes isolados
\end{itemize}

\vspace{0.5cm}

Pode-se realizar uma comunicação (síncrona) dentro de um grupo
\begin{itemize}
    \item Uma falha é detectada se um componente deixar de responder
\end{itemize}
    
\end{frame}

%% --------------------------------------------------------

\begin{frame}{Detecção de falhas utilizando grupos heterogêneos}

Grupos são formados por diferentes componentes
\begin{itemize}
    \item Grupos podem ser estáticos ou dinâmicos
    \item Cada grupo possui um coordenador
\end{itemize}

\vspace{0.5cm}

O coordenador do grupo é responsável por identificar falhas nos componentes de seu grupo
\begin{itemize}
    \item Comunicação síncrona
\end{itemize}

\vspace{0.5cm}

Mas o que acontece se o coordenador falhar?
\end{frame}

%% --------------------------------------------------------

\begin{frame}{Recuperação do sistema}

Após detectada a falha, um ponto importante é recuperar o funcionamento componente ou o sistema
\begin{itemize}
    \item Deve-se recuperar o estado correto
    \item Deve-se recuperar os dados atualizados
\end{itemize}

\vspace{0.5cm}

Pode-se fazer isso efetivamente de duas maneiras
\begin{itemize}
    \item Recuperação do componente (\textit{checkpoint})
    \item Atualização do componente
\end{itemize}
\end{frame}

%% --------------------------------------------------------

\begin{frame}{Recuperação do componente}

São utilizados \textit{checkpoints}
\begin{itemize}
    \item \textit{Backups} periódicos do estado do sistema
\end{itemize}

\vspace{0.5cm}

Quando uma falha é detectada, o componente é restaurado para o seu último \textit{backup} conhecido

\vspace{0.5cm}

Alto custo computacional (e talvez financeiro)
\begin{itemize}
    \item É necessário guardar diversos \textit{backups}
    \item Necessário espaço de armazenamento
    \item Muitas trocas de mensagens
    \item Hardware especializado é desejado
\end{itemize}
\end{frame}

%% --------------------------------------------------------

\begin{frame}{Atualização do componente}

Quando um componente falha, pode-se tentar atualizar seu estado
\begin{itemize}
    \item Novo estado considerado correto
    \item Neste novo estado, o componente pode continuar sua execução normalmente
\end{itemize}

\vspace{0.5cm}

Este processo é barato, mas difícil
\begin{itemize}
    \item Não é necessário realizar \textit{backups} ou adquirir novos itens de hardware
    \item Entretanto, é necessário saber exatamente qual erro ocorreu
\end{itemize}

\vspace{0.5cm}

Muitas vezes, só é realizada a reinicialização do componente
\begin{itemize}
    \item Espera-se que o melhor aconteça
\end{itemize}
\end{frame}

\end{document}
